\documentclass[12pt]{article}
\usepackage{amsmath}

\begin{document}

\begin{center}
\large\bf University of Waterloo\\
CS~466 --- Advanced Algorithm\\
Spring 2013\\
Problem Set 2\\
Siwei Yang - 20258568\\
\end{center}
\bigskip

\title{Problem Set 2}
\date{\today}
\author{Siwei Yang - 20258568}

\begin{enumerate}

\item{} [8 marks: The Need for Paid Exchanges]
In discussing Move to Front and other self-organizing heuristics the model was very important. In the model in which swapping the requested element with one immediately in front of it (and perhaps repeating this many times) as a “free exchange”, bring the question as to whether “paid exchanges” are necessary to achieve the offline optimal. “Clearly” they are not necessary to come within a factor of two of optimal under this model.

\begin{itemize}

\item[a.] Show that the offline optimal to service the request $a_{3},a_{2},a_{3},a_{2}$ is 8. The list is initially in the order $a_{1},a_{2},a_{3}$.

Assume we know the procedures to achieve offline optimal. Then, let $\delta_{0}, \delta_{1}, \delta_{2}, \delta_{3}$ represent the ordering before the before first, second, third and fourth accesses; let $c_{0}, c_{1}, c_{2}, c_{3}$ represent the cost for first, second, third and fourth accesses.

We know that $c_{0}, c_{1}, c_{2}, c_{3} \geq 1$. Assuming $c_{0} = 1$, then the access cost for $a_{2}$ in $\delta_{0}$ is greater than 1. Since $c_{1}$ is at least access cost for $a_{2}$ in $\delta_{0}$, we have $c_{1} \geq 2$. Thus,
\begin{equation} \label{eq:min-cost-1-2}
c_{0} + c_{1} \geq 3
\end{equation}

Otherwise  $c_{0}  \geq 2$, and $c_{1} \geq 1$, we still have equation~\ref{eq:min-cost-1-2}.

And for the same reason, we have
\begin{equation} \label{eq:min-cost-3-4}
c_{2} + c_{3} \geq 3
\end{equation}

Now, looking at the initial ordering $a_{1},a_{2},a_{3}$, if $\delta_{0}$ is $a_{1},a_{2},a_{3}$, then we have$c_{0} = 3$ and  $c_{1} \geq 2$. Thus, we have
\begin{equation} \label{eq:real-cost-initial}
c_{0} + c_{1} \geq 5
\end{equation}
which leads to
\begin{equation}
c_{0} + c_{1} + c_{2} + c_{3} \geq 8
\end{equation}
Thus, the \textbf{total cost is at least 8}.

Otherwise, $\delta_{0}$ is derivied from $a_{1},a_{2},a_{3}$ after exchanges. However, each paid exchange reduce access cost by one to at most one element. Therefore, we easily have
\begin{equation} \label{eq:real-cost-pay}
c_{0} + c_{1} \geq 4
\end{equation}
which leads to
\begin{equation}
1 + c_{0} + c_{1} + c_{2} + c_{3} \geq 8
\end{equation}
Thus, the \textbf{total cost is at least 8}.

if  $\delta_{0}$ is derivied from $a_{1},a_{2},a_{3}$ after one paid exchange. And, the \textbf{total cost is at least 8}. if  $\delta_{0}$ is derivied from $a_{1},a_{2},a_{3}$ after $k$ paid exchanges where $k \geq 2$, we still have
\begin{equation}
k + c_{0} + c_{1} + c_{2} + c_{3} \geq 8
\end{equation}
Thus, the \textbf{total cost is at least 8}.

From above, we conclude the optimal \textbf{total cost is at least 8}. And we actually achieve the total cost of 8 if we have $\delta_{0} = a_{2},a_{3},a_{1}$ is derivied from $a_{1},a_{2},a_{3}$ after two paid exchanges as well as $\delta_{0} = \delta_{1} = \delta_{2} = \delta_{3}$. Therefore, the optimal \textbf{total cost is 8}.

\item[b.] Also show that the optimal offline algorithm without using these “paid exchanges” is 9.

\end{itemize}

\medskip

\item{} [6 marks]
In a splay tree we use “double rotations” to move an element to the root. It was mentioned that moving the requested element to the root by a sequence can give very bad amortized behavior. Prove that this amortized cost can be $\Theta(n)$ (actually you can get it to n)for an arbitrarily long sequence of requests on a tree with n nodes, with a starting configuration of your choice. (You need not deal with either insertions or deletions.)

\end{enumerate}

\end{document}

