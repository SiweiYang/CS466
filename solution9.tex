\documentclass[12pt]{article}
\usepackage{amsmath}
\usepackage{amssymb}
\usepackage{mathtools}
\usepackage{tikz}

\begin{document}

\begin{center}
\large\bf University of Waterloo\\
CS~466 --- Advanced Algorithm\\
Spring 2013\\
Problem Set 9\\
Siwei Yang - 20258568\\
\end{center}
\bigskip

\begin{enumerate}

\item{} [14 marks: Mergesort]
Consider Mergesort when n is not (necessarily) a power of 2. The method works by (recursively) sorting a subarray of size $\frac{n}{2}$ and one of size $\frac{n}{2}$ and then merging them in n-1 comparisons. A segment of length 1 requires 0 comparisons.
\begin{itemize}
\item{(a)} [2 marks]
Give a recurrence relation that describes the number of comparisons used, in the worst case, by this method.
\begin{equation}
C(n) =
\begin{cases}
    C(\mathcal{b} \frac{n}{2} \mathcal{c}) + C(\mathcal{d} \frac{n}{2} \mathcal{e}) + n - 1,& \text{if } n > 1\\
    0,              & \text{otherwise}
\end{cases}
\end{equation}

\item{(b)} [4 marks]
Prove that n-1 comparisons are necessary (i.e. you cannot do it in fewer), in the \textit{worst case} for this merge step.

\item{(c)} [4 marks]
Prove that Mergesort, as described above, takes $n * \mathcal{d} \lg{n} \mathcal{e} - 2^{\mathcal{d} \lg{n} \mathcal{e}} + 1$ comparisons in the worst case.

\item{(d)} [4 marks]
The \textit{expected} number of comparisons for this method (over all possible permutations of the input) is a little $\theta(n)$ better. Prove it. (You do not have to deal with the exact constant in this $\theta(n)$ term.)

\end{itemize}


\end{enumerate}
\end{document}

