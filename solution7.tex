\documentclass[12pt]{article}
\usepackage{amsmath}
\usepackage{mathtools}
\usepackage{tikz}

\begin{document}

\begin{center}
\large\bf University of Waterloo\\
CS~466 --- Advanced Algorithm\\
Spring 2013\\
Problem Set 7\\
Siwei Yang - 20258568\\
\end{center}
\bigskip

\begin{enumerate}

\item{} [13 marks: Weighted Set Cover]
In class we looked at the set cover problem where we want a set cover of minimum number of subsets. A variant of the problem, which also has many applications, involves weights on the elements. The input consists of: a set U of n elements, where each u in U has a non-negative weight w(u); a collection $S_{1}, \ldots, S_{t}$ where $S_{i} \subset U$; and a number k. The problem is to pick k of the subsets Si to maximize the sum of the weights of the elements covered.

The goal of this question is to show that the obvious greedy algorithm has approximation factor $1-\frac{1}{e}$ (where e is the base of the natural logarithm). The greedy algorithm first picks a set that covers the maximum weight of elements, then deletes those elements and repeats until k sets have been chosen. Let $w^{*}$ be the weight of the elements covered by the optimum solution. Let $w_{i}$, for $i = 1, \ldots, k$, be the weight of the elements covered by the first i sets of the greedy algorithm.

Thus $w_{k}$ is the final weight of the greedy solution.

\begin{itemize}
\item{(a)} [2 marks]
Find an example with k = 2 and unit weights where $w_{2} = \frac{3}{4} * w^{*}$.


Let $U = \{1, 2, 3, 4\}$ and $S_{1} = \{1, 2\}$, $S_{2} = \{2, 3\}$, $S_{3} = \{3, 4\}$


By choosing $S_{2}$ at the first iteration and either $S_{1}, S_{3}$ at the second, we achieve a set cover of three elements. Thus weighting 3.
\begin{equation}
w_{2} = 3
\end{equation}
However the optimal solution is choosing $S_{1}$ and $S_{3}$ to produce a complete cover with weight of 4.
\begin{equation}
w^{*} = 4
\end{equation}

\item{(b)} [5 marks]
Prove that $w_{1} \geq \frac{w^{*}}{k}$. More generally, prove that $w_{i}-w_{i-1} \geq \frac{w^{*}-w_{i-1}}{k}$.


Assume without loss of generality:

\begin{itemize}
\item $U_{0}=U$ and $w_{0} = 0$
\item for the optimal algorithm, the subset chosen at iteration i is $S_{i}^{*}$.
\item for the greedy algorithm, that the working set at iteration i is $U_{i}$ and the subset will chosen at iteration i is $S_{i}$(so all subsets will be indexed).
\end{itemize}

Then, for the greedy algorithm, at any iteration i, we have the following:
\begin{equation} \label{eq:weight-against-all}
weight(S_{i} \cap U{i}) \geq weight(S_{j} \cap U{i}) \text{ where } j \geq i
\end{equation}
from the nature of greedy algorithm.

Then, for the optimal algorithm, we have:
\begin{equation} \label{eq:optimal-weight-divide}
weight(\bigcup_{1}^{k}S_{i}^{*} \cap U_{i}) + weight(\bigcup_{1}^{k}S_{i}^{*} \setminus U_{i}) = weight(\bigcup_{1}^{k}S_{i}^{*}) = w^{*}
\end{equation}

Consider the optimal k-set cover based off $U_{i}$, let its weight be $w^{*'}$ and subset chosen at iteration i be $S_{i}^{*'}$. Then:
\begin{equation}
weight(\bigcup_{1}^{k}S_{i}^{*'} \cap U_{i}) = w^{*'}
\end{equation}
Since all $weight(S_{i} \cap U{i}) \geq weight(S_{i}^{*'} \cap U{i})$ by ~\ref{eq:weight-against-all}
\begin{equation} \label{eq:optimal-sum}
k * weight(S_{i} \cap U{i}) \geq w^{*'}
\end{equation}

Now, let's observe the relation between $w^{*}$ and $w^{*'}$. Note that we can carefully choose $S_{i}^{*'}$s, so that all $S_{i}^{*}$s not chosen by the greedy algorithm is in $S_{i}^{*'}$s.
\begin{equation}
\bigcup_{j=1}^{k}S_{j}^{*} \subset (\bigcup_{j=1}^{i}S_{j} \cup \bigcup_{j=1}^{k}S_{j}^{*'}) = (U \setminus U_{i}) \cup \bigcup_{j=1}^{k}(S_{j}^{*'} \cap U_{i})
\end{equation}
Let the weight be  $w^{*''}$ for this instance. Then we have:
\begin{equation} \label{eq:optimal-against-optimal}
w_{i-1} + w^{*''} \geq w^{*}
\end{equation}

Since $w^{*'} \geq w^{*''}$ for optimality, we can combine ~\ref{eq:optimal-sum} and ~\ref{eq:optimal-against-optimal} to get:
\begin{equation} \label{eq:optimal-against-optimal}
k * weight(S_{i} \cap U{i}) \geq w^{*} - w_{i-1}
\end{equation}
where $weight(S_{i} \cap U{i}) = w_{i} - w_{i-1}$. Therefore, \textbf{for every iteration i, we have $w_{i}-w_{i-1} \geq \frac{w^{*}-w_{i-1}}{k}$} which resolves to $w_{1} \geq \frac{w^{*}}{k}$ when $i = 1$.

\item{(c)} [6 marks]
Prove by induction that $w_{i} \geq (1-(1-\frac{1}{k})^{i}) * w^{*}$.

\begin{itemize}
\item \textbf{Base case}: $w_{1} \geq \frac{w^{*}}{k} = (1-(1-\frac{1}{k})^{1}) * w^{*}$
\item \textbf{Induction}: Assume we have $w_{i} \geq (1-(1-\frac{1}{k})^{i}) * w^{*}$
Combined with $w_{i+1}-w_{i} \geq \frac{w^{*}-w_{i}}{k}$, we have:
\begin{align*}
w_{i+1} &\geq \frac{w^{*}}{k} + (1 - \frac{1}{k}) * w_{i}
\\
&\geq \frac{w^{*}}{k} + (1 - \frac{1}{k}) * (1-(1-\frac{1}{k})^{i}) * w^{*}
\\
&= \frac{w^{*}}{k} + (1 - \frac{1}{k}) * (w^{*} - (1-\frac{1}{k})^{i} * w^{*})
\\
&= w^{*} - (1-\frac{1}{k})^{i+1} * w^{*}
\\
&= (1-(1-\frac{1}{k})^{i+1}) * w^{*}
\end{align*}
Therefore, we have \textbf{$w_{i+1} \geq (1-(1-\frac{1}{k})^{i+1}) * w^{*}$}.
\item \textbf{Conclusion}: \textbf{$w_{i} \geq (1-(1-\frac{1}{k})^{i}) * w^{*}$ stands for all $i > 0$}.
\end{itemize}

\item{(d)} [0 marks]
From part (c), the weight of the greedy solution, $w_{k}$, is at least $(1-(1-\frac{1}{k})^{k}) * w^{*}$. Since $\lim_{k \to \infty} 1-(1-\frac{1}{k})^{k} = 1 - \frac{1}{e}$ and $1-(1-\frac{1}{k})^{k}$ is decreasing, thus $1-(1-\frac{1}{k})^{k} \geq 1 - \frac{1}{e}$, so the approximation ratio of the greedy algorithm is $1 - \frac{1}{e}$. In fact $1 - \frac{1}{e}$ is the best approximation factor possible for this problem (unless P=NP).

\end{itemize}


\end{enumerate}

\end{document}

